\documentclass[conference]{Template/IEEEtran}
\IEEEoverridecommandlockouts
\usepackage{cite}
\usepackage{amsmath,amssymb,amsfonts}
\usepackage{algorithmic}
\usepackage{graphicx}
\usepackage{textcomp}
\usepackage{xcolor}
\usepackage{float}
\usepackage{subfigure}
\usepackage{steinmetz}
\usepackage [hidelinks, breaklinks=true] {hyperref} 
\usepackage{float}
\usepackage{color}
\usepackage{tikz}
\usepackage{subfigure} 
\usepackage{pdfpages}
\usepackage{fancyhdr}
\usepackage{textcomp} %para simbolo de marca registrada
\def\BibTeX{{\rm B\kern-.05em{\sc i\kern-.025em b}\kern-.08em
    T\kern-.1667em\lower.7ex\hbox{E}\kern-.125emX}}
    
   \graphicspath{Figuras/ }

%Configuración del pie de página. Se puede hacer con fancyfoot o por cada bloque de pie separado (izq, derecho y central; lo mismo para la cabecera) . Si no se deja en blanco cfoot{} el numero de pagina se pone al medio por default y pisa las letras de las otras cosas. 

\pagestyle{fancy}
%\fancyfoot[L]{}
\lfoot{\textsc{Universidad Tecnológica Nacional, Facultad Regional Córdoba - Argentina}}
\cfoot{} % quitar número de página del centro
\rfoot{\thepage} % número de página a la derecha
\renewcommand{\headrulewidth}{0pt} % grosor de la línea de la cabecera
\renewcommand{\footrulewidth}{0pt} % grosor de la línea del pie

  %foot layout 
\pagestyle{fancy}
\title{Diseño de conversor Buck-Boost para celdas fotovoltaicas con MPPT} 
\author{
\IEEEauthorblockN{ Matias Dogliani}
\IEEEauthorblockA{\textit{Universidad Tecnológica Nacional, Departamento de Ingeniería Electrónica} \\
\textit{Electrónica de Potencia} \\
2020} \\

}



    %Tittle and author layout 

\begin{document}
\maketitle\thispagestyle{fancy}
\begin{abstract}


    In this paper a low-cost and simple DC-DC Buck-Boost converter is designed for photo-voltaic (PV) aplicattions, for Maximum Power Point Tracking integration for variable voltage input and variable load (i.e charging a battery bank system) suitable for residencial use. 
    

\end{abstract}
\begin{resumen}
   Se presenta el diseño de un coversor DC-DC de bajo costo y alta simplicidad para su integración en controladores MPPT de sistemas de paneles solares para uso doméstico. 
\end{resumen}

\begin{IEEEkeywords}
   Dc-dc converter, MPPT, PV, Solar, buck-boost
\end{IEEEkeywords}

%Cambiar abstract. Darle enfioque domiciliaroi  

\section{Introducción}

    En los últimos años se ha presenciado el auge de las energías renovables, principalmente por el aumento de la demanda de energía a nivel mundial conjuntamente con el aumento del costo de la misma y el altamente negativo impacto ambiental del uso de la energía convencional (fósil) \cite{gallardo2014diseno}. La energía solar, perteneciente a las energías denominadas limpias, puede ser utilizada fácilmente en sistemas de energía  residenciales y/o industriales, particularmente en áreas rurales o remotas. 
    
    En las interfaces utilizadas entre las celdas fotovoltaicas y la carga domiciliaria (cargadores de baterías, luminaria DC, etc) podemos encontrar conversores de bajo costo pero así de también de una eficiencia reducida dado que el punto de operación de los paneles solares, en la mayoría de los casos, no es adecuado \cite{Chang}. Distintos tipos de conversores DC-DC son utilizados, donde cada uno presenta cierta deficiencia y o limitación en cuanto a la regulación de voltaje de entrada y salida, valor de carga, etc. Además el control de estos conversores de bajo costo se realiza con técnicas basadas en PWM. 
    
    Considerando las características de los paneles PV, el voltaje de salida de los mismos varia ampliamente de acuerdo a la temperatura ambiente, radiación solar, tiempo de sombra, y otras condiciones climáticas como así también de parámetros inherentes de los mismos como conexión de celdas, cantidad de celdas,etc. Por lo que para lograr un voltaje estable y lograr una un punto de máxima potencia de trabajo, un conversor buck-boost se coloca entre los panales solares y las cargas de continua (i.e banco de baterías) y un control de rastreo de máximo punto de potencia \cite{Chang} como puede observarse en la imagen (\ref{fig: Curva MPPT}), que pertenece al panel solar \textregistered Siemens modelo SM50-H
    
        \begin{figure}[htbp]
            \centering
            \includegraphics[scale = 0.25]{Figuras/Curva_MPPT.png}
            \caption{$I_{PV}$ y $P_{PV}$ vs $V_{PV}$ \cite{gallardo2014diseno}  }
            \label{fig: Curva MPPT}
        \end{figure}
    
    Existen diversas topologías para el conversor propuesto. En \cite{Kiran} se propone un conversor con un solo transistor y  de \textit{soft switching}, en \cite{9106477} un circuito de alta eficiencia, de diseño complejo por la cantidad de inductores; en \cite{5701795} se propone un conversor de topología convencional junto con su controlador analógico para una variación de entrada determinada; y en \cite{6119227} un conversor que combina topología KY y Buck para lograr un conversor estable. En \cite{954206} se puede observar comparaciones entre algunas topologías. En este trabajo se propone un conversor de simple implementación, bajo costo y eficiencia aceptable, con modo de funcionamiento de conducción continua, para ser controlado a través de un microcontrolador como se realiza en \cite{Chang}. De esta forma, en conjunto con sensores de corrientes y tensión, se podrá lograr un conversor y regulador para cargas variables, y configuraciones de paneles diferentes, configurables a través de una interfaz.
 
    La topología propuesta se presenta en la sección de diseño junto con un análisis general del mismo. En la sección de determinación de componentes se contempla las condiciones de trabajo para un caso determinado típico y expresiones para encontrar los valores críticos de componentes pasivos. En resultado, finalmente, se muestran las curvas de tensión de entrada y salida. 
 
\section{Diseño}
    
    Se propone un conversor de tipo SEPIC \textit{single-ended primary-inductor converter} ya que posee, entre otras ventajas, ausencia de inversión de polaridad en la salida, bajo rizado de la corriente de entrada \cite{alvarez2019caracterizacion} y la utilización de un solo transistor MOSFET que facilitará la implementación del controlador en trabajos futuros. El circuito es el de la figura \ref{fig: Circuito sepic}
    \begin{figure}
        \centering
         \includegraphics[scale = 0.25]{Figuras/Circuito_BB.png}
        \caption{Circuito sepic}
        \label{fig: Circuito sepic}
    \end{figure}
       
      
    En su mayoría, las diferentes topologías siguen un mismo procedimiento de diseño. Este trata de analizar al  conversor en sus dos modos de funcionamiento por separado, modo elevador y modo reductor de tensión. El diseño se realiza con este procedimiento mencionado, siguiendo lineamientos de \cite{espinosa2017asynchronous} y recomendaciones de notas de aplicación \cite{haifengdesign} y \cite{falin2008designing} considerando al conversor en modo de conducción continua (CCM). 
    
    
    Cuando el transistor Q1 se encuentra encendido (ON) el diodo se encuentra en estado de bloqueo (sin importar el valor de $V_{in}$ o $V_{out}$, en este estado, la tensión en el diodo es negativa) y el capacitor $C_p$ se carga a la tensión de entrada y la tensión en $L_2$ es la misma pero de polaridad invertida. Cuanto el transistor Q1 se encuentra apagado (OFF) el diodo comienza a conducir manteniendo la continuidad de corrientes en los inductores. La tensión en el inductor $L_2 = L_{1b}$ es igual a la tensión de salida invertida $-V_{out}$ (considerando la caída en el diodo como nula) y la tensión el inductor $L_1 = L_{1a}$ será la tensión $V_{in} - V_{C_p} - V_{out}$. Siguiendo estos modos se pueden obtener las curvas de tensión y corriente de cada elemento, como se representa de forma aproximada en la figura 
  
    \begin{figure}[H]
            \centering
          \includegraphics[scale = 0.25]{Figuras/Modos_BB.png}
            \caption{Circuitos equivalentes para cada modo \cite{falin2008designing}}
            \label{fig: Circuito sepic}
      \end{figure}
       
     \begin{figure}[htbp]
            \centering
             \includegraphics[scale = 0.25]{Figuras/curvas_BB.png}
            \caption{Curvas de corriente y tensión}
            \label{fig: Circuito sepic}
      \end{figure}
  
    Considerando a los elementos como ideales, la función de transferencia del circuito está definida por la expresión \ref{ec: funcion de transferencia}
    
    \begin{equation}
        \frac{V_{out}}{V_{in}} = \frac{D}{1-D} = M 
        \label{ec: funcion de transferencia}
    \end{equation}
    
    \section{Determinación de elementos pasivos}
    
    Para el diseño se consideran rangos determinados típicos de paneles solares utilizados y como carga un banco de batería de 12~V, aunque cabe destacar que el circuito será controlado para conservar la eficiencia de estas condiciones aún cuando las mismas sean diferentes en pequeña medida, factor importante para no olvidar que la selección de los componentes debe realizarse considerando condiciones de plena carga y máxima potencia de transferencia. Se considera el panel mencionado \cite{PV_datasheet}. 
        \subsection{Ciclo de trabajo}
        
         Considerando un rango de variación de irradiación recibida entre $20\%$ y $100\%$  la potencia producida por las celdas estará entre 10~W y 50~W aproximadamente, y la tensión de entrada variará entre 8 y 16. Y la corriente del punto máximo de potencia variará en la misma medida entre 0.63~A y 3.15~A. La resistencia de entrada mínima y máxima estarán determinadas por estos valores, que definirán la ganancia necesaria del circuito \cite{atia2009photovoltaic}
 
            \begin{align}
                M_{min} &= \sqrt{\frac{R_L}{R_{in_{M}}}} \\
                M_{max} &= \sqrt{\frac{R_L}{R_{in_{m}}}}
            \end{align}
       
        El ciclo de trabajo mínimo se calculará según la expresión \ref{ec: D minimo} , y el ciclo de trabajo máximo según \ref{ec: D maximo} . Se puede calcular solo el ciclo de trabajo máximo, que ocurre cuando la tensión de entrada es mínima, según la simple expresión de \cite{falin2008designing} teniendo en cuenta la función de transferencia del circuito. 
        
        $D_{max} = \frac{12~V +0.6V}{8 +0.6 + 12} = 0.61$
           
            \begin{align}
                D_{min} &= \frac{M_{min}}{ 1 + M_{min}} \label{ec: D minimo} \\
                D_{max} &= \frac{M_{max}}{ 1 + M_{max}} \label{ec: D maximo}
            \end{align}
      
        
  
        \subsection{Inductor}
        
            Los inductores $L_1$ y $L_2$ están relacionado con la frecuencia de conmutación del transistor. Para poder implementar un controlador MPPT es necesario cuidar las posibles oscilaciones alrededor del punto óptimo, por lo que es recomendado optar por un ripple de corriente relativamente pequeño \cite{atia2009photovoltaic} ($2\%$) de la mayor corriente. La decisión del factor de ripple que se permitirá es un factor determinante para la elección de todos los componentes y una regla práctica es considerarlo entre un $20$ y $40\% $ \cite{falin2008designing}. Se considera un $30\%$ de la corriente de entrada efectiva \cite{kaouane2015design}. 
            
            El valor de los mismos se calcula con la ecuación \ref{ec: Calculo de inductores}: 
            
            $L_1 = L_2 =23~\mathrm{mH}$
            
            \begin{equation}
                L_1 = L_2 = \frac{V_{in_{min}}}{2\Delta I_L \cdot f_s} \times D_{max} 
                \label{ec: Calculo de inductores}
            \end{equation}
            
            Se considera inductores acoplados entre ellos ya que esto mejora aún más la eficiencia del circuito \cite{rios2017sistema}
    
        \subsection{Capacitores}
            
            Cuando el transistor Q1 está encendido el capacitor de salida debe proveer a la carga la corriente necesaria. Este debe contar con la suficiente capacitancia y la menor posible ESR, para cumplir con los requerimientos del \textit{ripple} de voltaje de salida, cuidando también que soporte la corriente de salida máxima. 
            
            Para calcular el capacitor de salida es válido considerar al circuito como un conversor \textit{boost} convencional (ya que es idéntica la etapa de salida). De esta forma, utilizando la expresión \ref{ec : Capacitor de salida }\cite{falin2008designing}, se obtiene aproximadamente: 
            
            $C_{out} = 122~\mu F$
            
            \begin{equation}
                C_{out} \geq \frac{I_{out}D_{max}T_s}{\Delta V_{out}}
                \label{ec : Capacitor de salida }
            \end{equation}
        
            El capacitor de entrada y de aislación $C_p$ no poseen un valor crítico, solo se debe cuidar que puedan soportar potencia que deba circular a través de ellos. Y lo mismo con el diodo, debiendo soportar además el máximo voltaje en reversa y los picos repetitivos de tensión.
        
        
        
\section{Resultados}
 
 
    Con los componentes antes calculados se realizaron una serie de simulaciones para comprobar el comportamiento del circuito. El circuito que se utilizó en la simulación es el de la figura \ref{fig: Circuito simulación}. 
    
     \begin{figure}[htbp]
            \centering
             \includegraphics[scale = 0.3]{Figuras/Circuito_simulacion.png}
            \caption{Simulación en LTSpice}
            \label{fig: Circuito simulación}
      \end{figure}
    
    Para una tensión de entrada $V_{i_min}$ se puede observar en las curvas de la figura \ref{fig:Curvas Vin9V}  que se obtiene una salida $V_{out} \approx 12~V$ utilizando el ciclo de trabajo $D_{max} = 0.61$ calculado anteriormente. Esto comprueba la capacidad del circuito en mantener una tensión de salida al nivel deseado. 
    
    \begin{figure}[htbp]
            \centering
             \includegraphics[scale = 0.3]{Figuras/Curvas_Vin9V.png}
            \caption{$V_{in_{min}}$, $V_{out}$ y $D_{max}$}
            \label{fig:Curvas Vin9V}
      \end{figure}
    
    En la figura se puede apreciar el ripple de la tensión de salida, para la entrada de tensión mínima. Este valor logrado es totalmente aceptable
    
    \begin{figure}[htbp]
            \centering
             \includegraphics[scale = 0.3]{Figuras/rippleVin9.png}
            \caption{ Ripple de salida $\Delta V_{out}$ }
            \label{fig: Ripple de salida}
      \end{figure}
    
    
   Para una tensión de entrada máxima para el caso particular considerado se puede observar las curvas de la figura \ref{fig:Curvas Vin16V}  .En este caso también el ripple de la tensión de salida es acorde al valor esperado según el diseño (fig. \ref{fig: Ripple de salida VinMax})
   
    \begin{figure}[htbp]
            \centering
             \includegraphics[scale = 0.3]{Figuras/Curvas_Vin16V.png}
            \caption{$V_{in_{max}}$, $V_{out}$ y $D_{max}$}
            \label{fig:Curvas Vin16V}
      \end{figure}
      
      \begin{figure}[htbp]
            \centering
             \includegraphics[scale = 0.3]{Figuras/rippleVin16V.png}
            \caption{ Ripple de salida $\Delta V_{out}$ con $V_{in_{max}}$ }
            \label{fig: Ripple de salida VinMax}
      \end{figure}
   
   
    El comportamiento dinámico de este circuito es pobre (fig. ). Posee un tiempo de crecimiento relativamente grande. Esto se podrá mejorar en trabajos futuros pero se buscará siempre cuidar la estabilidad del mismo y no priorizar la rapidez, ya que las condiciones de trabajo del panel solar no cambian de forma drástica (aunque sí lo puede hacer la carga). 
    
    \begin{figure}[htbp]
            \centering
             \includegraphics[scale = 0.3]{Figuras/Transitorio_Vin9V.png}
            \caption{ Transitorio de tensión de salida }
            \label{fig: transitorio Vout}
      \end{figure}
   
   
   Al cambiar la carga el transitorio aumenta al igual que el ripple en la tensión de salida como se observa en la figura \ref{fig: transitorio Vout carga menor}
   
   
     \begin{figure}[htbp]
            \centering
             \includegraphics[scale = 0.27]{Figuras/Vout_R1.png}
            \caption{ Transitorio de tensión de salida con carga menor }
            \label{fig: transitorio Vout carga menor}
      \end{figure}
   
   Este circuito es parte de un sistema de mayor complejidad como el de la figura \ref{fig: PVControl} en donde se aplicará un control basado en el algoritmo MPPT para obtener una gran eficiencia de transferencia de potencia. Es por ello que se eligió esta topología por ser de fácil control. 
   
    \begin{figure}[htbp]
            \centering
             \includegraphics[scale = 0.35]{Figuras/Diagrama1.png}
            \caption{ Regulador para paneles solares}
            \label{fig: PVControl}
      \end{figure}

    
 
    

\section{Conclusión}


    Se diseñó un conversor Buck-Boost de topología SEPIC de alta eficiencia y bajo costo en comparación con las distintas topologías mencionadas. La fortaleza de este diseño es la facilidad con la que podrá realizarse el sistema de control MPPT y la integración e implementación del mismo. 
    Este conversor presenta un aislamiento entre entrada y salida y un nivel de rizado de corriente  y tensión aceptable que puede ser mejorado si los bobinados son enrollados bajo un mismo núcleo. 

    De los resultados se puede concluir que este conversor es apto para aplicaciones de energía renovable y carga de DC como banco de baterías.  Se propone como trabajo a futuro a mejorar el diseño cuidando las interferencias producidas por este tipo de conversores conmutados. 



\nocite{*}
\bibliographystyle{plain}
\bibliography{Referencia/ref.bib}

\end{document}

